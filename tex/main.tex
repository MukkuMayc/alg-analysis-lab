\documentclass[a4paper,12pt]{article}

\usepackage[T2A]{fontenc}
\usepackage[utf8]{inputenc}
\usepackage[russian]{babel}

\usepackage{color}   %May be necessary if you want to color links
\usepackage{hyperref}
\hypersetup{
    colorlinks=true, %set true if you want colored links
    linktoc=all,     %set to all if you want both sections and subsections linked
    linkcolor=black, %choose some color if you want links to stand out
    citecolor=black,
}

\usepackage[nottoc,numbib]{tocbibind} %for adding bibliography to 
                                      %table of content

% \usepackage{algpseudocode}
\usepackage{adjustbox}
\usepackage{csvsimple}

\usepackage{graphicx} % for images

\usepackage{float} % for floating figure

\usepackage[noend]{algorithmic}
\newcommand{\TITLE}[1]{\item[#1]}
\renewcommand{\algorithmiccomment}[1]{$/\!/$ \parbox[t]{4.5cm}{\raggedright #1}}
% ugly hack for for/while
\newbox\fixbox
\renewcommand{\algorithmicdo}{\setbox\fixbox\hbox{\ {} }\hskip-\wd\fixbox}
% end of hack
\newcommand{\algcost}[2]{\strut\hfill\makebox[1.5cm][l]{#1}\makebox[4cm][l]{#2}}

\begin{document}

\title{Эмпирический анализ алгоритма Беллмана-Форда}
\author{Козырев Сергей, 331 группа}
\maketitle

\newpage
\tableofcontents

\newpage
\section{Краткое описание алгоритма}
В книге Кормена "Алгоритмы: построение и анализ"\hspace{1mm}даётся следующее 
описание\cite[с.~688]{cormen}:
\begin{quotation}
Алгоритм Беллмана-Форда (Bellman-Ford algorithm) решает задачу о кратчайшем 
пути из одной вершины в общем случае, когда вес каждого из рёбер может быть 
отрицательным.
Для заданного взвешенного ориентированного графа $G = (V, E)$ с
истоком $s$ и весовой функцией $w : E \rightarrow R$ алгоритм Беллмана-Форда 
возвращает логическое значение, указывающее, содержится ли в графе цикл с 
отрицательным весом, достижимый из истока.
Если такой цикл существует, алгоритм указывает, что решения не существует.
Если же таких циклов нет, алгоритм выдаёт кратчайшие пути и их
веса.
\end{quotation}

Алгоритм впервые был предложен Альфонсом Шимбелем (1955), но вместо него был 
назван в честь Ричарда Беллмана и Лестера Форда, которые опубликовали его в 
1958 и 1956 соответственно.
Эдвард Мур также опубликовал его в 1957 году.\cite[с.~32-33]{schrijver}

Этот алгоритм использует, например, протокол маршрутной информации 
(RIP).\cite{semenov}

Дадим описание алгоритма в псевдокоде.
Будем использовать две вспомогательные функции: 
Initialize-Single-Source и Relax.
Псевдокод взят из \cite[с.~686,~689]{cormen}.
\begin{algorithmic}[1]
  \TITLE{\textsc{Initialize-Single-Source}$(G, s)$}
  \FOR{каждой вершины $v \in G.V$}
    \STATE $v.d = \infty$
    \STATE $v.\pi =$ NIL
  \ENDFOR
  \STATE $s.d = 0$
\end{algorithmic}
\begin{algorithmic}[1]
  \TITLE{\textsc{Relax}$(u, v, w)$}
  \IF{$v.d > u.d + w(u, v)$}
    \STATE $v.d = u.d + w(u, v)$
    \STATE $v.\pi = u$
  \ENDIF
\end{algorithmic}
\newpage
\begin{algorithmic}[1]
  \TITLE{\textsc{Bellman-Ford}$(G, w, s)$}
  \STATE \textsc{Initialize-Single-Source}$(G, s)$
  \FOR{$i = 1$ to $|G.V| - 1$}
    \FOR{каждого ребра $(u, v) \in G.E$}
      \STATE \textsc{Relax}$(u, v, w)$
    \ENDFOR
  \ENDFOR
  \FOR{каждого ребра $(u, v) \in G.E$}
    \IF{$v.d > u.d + w(u, v)$}
      \RETURN FALSE
    \ENDIF
  \ENDFOR
  \RETURN TRUE
\end{algorithmic}

\section{Математический анализ алгоритма}
  Время работы алгоритма Беллмана-Форда составляет $O(V\,E)$, поскольку
  инициализация в строке 1 занимает время $\Theta(V)$, на каждый из $|V| - 1$ 
  проходов по рёбрам в строках 2--4 требуется время $\Theta(E)$, а на выполнение цикла
  \texttt{for} в строках 5--7 затрачивается время $O(E)$.\cite[с.~689-690]{cormen}

\section{Характеристики входных данных и их генератора}
На вход алгоритму поступает граф и индекс вершины, от которой ищем пути.
Граф задаётся как вектор вершин и вектор рёбер.
Вершины хранят вес пути от них до исходной вершины и предыдущую в этом пути вершину.
Рёбра задаются черёз упорядоченную тройку: начало, конец и вес.

Генератор создаёт граф по заданному количеству вершин и рёбер.
Сначала он создаёт вектор вершин, потом для каждого ребра случайным образом 
выбирает начало, конец и вес.
В результате получается взвешенный направленный псевдограф с петлями. \\
Код генератора: \url{https://github.com/MukkuMayc/alg-analysis-lab/blob/master/src/generator.cpp}

Чтобы сгенерировать данные о времени выполнения, 
на вход подаём количество вершин и рёбер, по ним генерируем граф 
и замеряем время выполнения алгоритма.
Единица измерения трудоёмкости - время(в наносекундах). Для замера использовался Google Benchmark.

Было проведено два эксперимента:
\begin{itemize}
  \item Количество вершин меняется от 100 до 1600, увеличиваясь в 2 раза.
  Количество рёбер, аналогично, меняется от 3200 до 102400, увеличиваясь в 2 раза.
  Для каждой пары параметров алгоритм запускался 10 раз.
  В данном эксперименте мы рассматриваем изменение времени при удвоении данных.
  \item 
  Количество вершин меняется от 1000 до 2000 включительно с шагом в 100.
  Количество рёбер меняется от 4000 до 5000 включительно с шагом в 100.
  Для каждой пары параметров алгоритм запускался 100 раз,
  после чего данные усреднялись и полученный результат брался за
  значение при этих параметрах.
  По результатам данного эксперимента мы строим график и линейную модель (Рис. \ref{data_and_regression})
\end{itemize}

\section{Реализация алгоритма}
Алгоритм:
\url{https://github.com/MukkuMayc/alg-analysis-lab/blob/master/src/bellman-ford.cpp} \\
Здесь собирается статистика:
\url{https://github.com/MukkuMayc/alg-analysis-lab/blob/master/src/benchmark/benchmark.cpp}
\section{Полученные результаты и их анализ}
В таблице \ref{tbl:data1} приведены полученные в первом эксперименте данные. \\
В таблице \ref{tbl:data1_ratio} значения вычислялись на основе значений таблицы \ref{tbl:data1} следующим образом
($a_{ij}$ - элемент i-й строки, j-го столбца таблицы \ref{tbl:data1_ratio}, $b_{ij}$ - элемент таблицы \ref{tbl:data1}):
\begin{itemize}
  \item если у $b_{ij}$ есть ячейки выше и левее, то 
  $a_{ij} = \frac12 \left(\frac{b_{ij}}{b_{i-1, j}} + \frac{b_{ij}}{b_{i, j-1}}\right)$
  \item если у $b_{ij}$ есть только ячейка выше, то
  $a_{ij} = \frac{b_{ij}}{b_{i-1, j}}$ 
  \item если у $b_{ij}$ есть только ячейка левее, то
  $a_{ij} = \frac{b_{ij}}{b_{i, j-1}}$
  \item в ином случае, $a_{ij} = 0$
\end{itemize}
\newpage
\begin{center}
  \begin{table}[h]
    \centering
    \scalebox{0.9}{
      \csvautotabular{data1.csv}{}% use head of csv as column names
    }
    \caption{Время выполнения в первом эксперименте, нc}
    \label{tbl:data1}
  \end{table}
  \begin{table}[h]
    \centering
    \scalebox{0.9}{
      \csvautotabular{data1_diff.csv}{}% use head of csv as column names
    }
    \caption{Смысл содержимого описан выше}
    \label{tbl:data1_ratio}
  \end{table}
\end{center}

Здесь видно, что при удвоении данных время также удваивается.
Действительно, это совпадает с теоретическим соотношением:
\[
  \frac{T(2n)}{T(n)} = \frac{2 (c E V)}{c E V} = 2, \textrm{V - количество вершин, E - количество рёбер}, c = const
\]

Теперь перейдём ко второму эксперименту. \\
Его результаты находятся здесь:
\url{https://github.com/MukkuMayc/alg-analysis-lab/blob/master/tex/data2.csv} \\
По ним был построен график (Рис. \ref{data_and_regression}). По оси X находится произведение $E \cdot V$, где E - количество рёбер, а V - количество вершин.
По оси Y - время выполнения. Также был построен график линейной регрессии.

\begin{figure}[H]
  \centering
  \def\svgwidth{\columnwidth}
  \input{data_and_regression.pdf_tex}
  \caption{Данные второго эксперимента и линейная регрессия}
  \label{data_and_regression}
\end{figure}

Мы видим, что данные соотносятся с приведённой выше оценкой сложности алгоритма,
время прямо пропорционально величине $E \cdot V$.
По каждому параметру в отдельности зависимость линейная, это видно в первом эксперименте.
\begin{thebibliography}{9}
  \bibitem{cormen}
    Томас Х.Кормен, Чарльз И.Лейзерсон, Рональд Л.Ривест, Клиффорд Штайн
    \textit{Алгоритмы: построение и анализ}
    - 3-е изд - М.: Издательский дом «Вильямс», 2013. — ISBN 978-5-8459-1794-2.
  \bibitem{schrijver}
    Schrijver, Alexander (2005).
    \href{https://homepages.cwi.nl/~lex/files/histco.pdf}
    {"On the history of combinatorial optimization (till 1960)"} (PDF).
    \textit{Handbook of Discrete Optimization.} Elsevier: 1–68.
  \bibitem{semenov}
    Семёнов Ю.А.,
    \href{http://citforum.ru/nets/semenov/4/44/rip44111.shtml}{Протокол маршрутизации RIP}
    / Телекоммуникационные технологии, 2004
\end{thebibliography}
\section{Характеристики использованной вычислительной среды и оборудования}
ОС: Arch Linux x86\_64 \\
Ядро: 5.3.13-arch1-1 \\
ЦП: Intel i3-3240 (4) @ 3.400GHz \\
Компилятор: clang version 9.0.0

\end{document}