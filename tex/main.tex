\documentclass[a4paper,12pt]{article}

\usepackage[T2A]{fontenc}
\usepackage[utf8]{inputenc}
\usepackage[russian]{babel}

\usepackage{color}   %May be necessary if you want to color links
\usepackage{hyperref}
\hypersetup{
    colorlinks=true, %set true if you want colored links
    linktoc=all,     %set to all if you want both sections and subsections linked
    linkcolor=black, %choose some color if you want links to stand out
    citecolor=black,
}

\usepackage[nottoc,numbib]{tocbibind} %for adding bibliography to 
                                      %table of content

\usepackage{algpseudocode}

\begin{document}

\title{Эмпирический анализ алгоритма Форда-Беллмана}
\author{Козырев Сергей, 331 группа}
\maketitle
\newpage
\tableofcontents

\newpage
\section{Краткое описание алгоритма}
Алгоритм Беллмана-Форда (Bellman-Ford algorithm) решает задачу о кратчайшем 
пути из одной вершины в общем случае, когда вес каждого из ребер может быть 
отрицательным. Для заданного взвешенного ориентированного графа $G = (V, E)$ с 
истоком $s$ и весовой функцией $w : E \rightarrow R$ алгоритм Беллмана-Форда 
возвращает логическое значение, указывающее, содержится ли в графе цикл с 
отрицательным весом, достижимый из истока. Если такой цикл существует, алгоритм 
указывает, что решения не существует. Если же таких циклов нет, алгоритм выдаёт 
кратчийшие пути и их веса. \cite{cormen}

Алгоритм впервые был предложен Альфонсом Шимбелем (1955), но вместо него был 
назван в честь Ричарда Беллмана и Лестера Форда, которые опубликовали его в 
1958 и 1956 соответственно. Эдвард Мур также опубликовал этот алгоритм в 
1957 году. \cite{schrijver}

Дадим описание алгоритма в псевдокоде. Но сначала зададим две вспомогательные 
функции.
\begin{algorithmic}
  \Function{Initialize-Single-Source}{$G$, $s$}
    \For{каждой вершины $v \in G.V$}
      \State $v.d = \infty$
      \State $v.\pi =$ NIL
    \EndFor
    \State $s.d =$ 0
  \EndFunction

  \Function{Relax}{$u$, $v$, $w$}
    \If{$v.d > u.d + w(u, v)$}
      \State $v.d = u.d + w(u, v)$
      \State $v.\pi = u$
    \EndIf
  \EndFunction
\end{algorithmic}
А вот и сам алгоритм
\begin{algorithmic}
  \Function{Bellman-Ford}{$G$, $w$, $s$}
  \State \Call{Initialize-Single-Source}{$G$, $s$}
    \For{$i = 1$ to $|G.V| - 1$}
      \For{каждого ребра $(u, v) \in G.E$}
        \State \Call{Relax}{u, v, w}
      \EndFor
    \EndFor
    \For{каждого ребра $(u, v) \in G.E$}
      \If{$u.d > u.d + w(u, v)$}
        \State \Return FALSE
      \EndIf
    \EndFor
    \State \Return TRUE
  \EndFunction
\end{algorithmic}

\section{Математический анализ алгоритма}
  Время работы алгоритма Беллмана-Форда составляет $O(V\,E)$, поскольку
  инициализация в строке 1 занимает время $\Theta(V)$, на каждый из $|V| - 1$ 
  проходов по рёбрам в строках 2-4 требуется время $\Theta(E)$, а на выполнение цикла 
  \texttt{for} в строках 5-7 затрачивается время $O(E)$.\cite{cormen}
\section{Характеристики входных данных и их генератора}
\section{Реализация алгоритма}
\section{Полученные результаты и их анализ}
\begin{thebibliography}{9}
  \bibitem{cormen}
    Томас Х. Кормен, Чарльз И. Лейзерсон, Рональд Л. Ривест, Клиффорд Штайн
    \textit{Алгоритмы: построение и анализ}
    - 3-е изд - М.: Издательский дом «Вильямс», 2013. — ISBN 978-5-8459-1794-2.
  \bibitem{schrijver}
    Schrijver, Alexander (2005). 
    \href{https://homepages.cwi.nl/~lex/files/histco.pdf}
    {"On the history of combinatorial optimization (till 1960)"} (PDF).
    \textit{Handbook of Discrete Optimization.} Elsevier: 1–68.
\end{thebibliography}
\section{Характеристики использованной вычислительной среды и оборудования}
\end{document}